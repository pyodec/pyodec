% Generated by Sphinx.
\def\sphinxdocclass{report}
\documentclass[letterpaper,10pt,english]{sphinxmanual}
\usepackage[utf8]{inputenc}
\DeclareUnicodeCharacter{00A0}{\nobreakspace}
\usepackage{cmap}
\usepackage[T1]{fontenc}
\usepackage{babel}
\usepackage{times}
\usepackage[Bjarne]{fncychap}
\usepackage{longtable}
\usepackage{sphinx}
\usepackage{multirow}


\title{Pyodec Documentation}
\date{July 11, 2014}
\release{0.0}
\author{Joe Young}
\newcommand{\sphinxlogo}{}
\renewcommand{\releasename}{Release}
\makeindex

\makeatletter
\def\PYG@reset{\let\PYG@it=\relax \let\PYG@bf=\relax%
    \let\PYG@ul=\relax \let\PYG@tc=\relax%
    \let\PYG@bc=\relax \let\PYG@ff=\relax}
\def\PYG@tok#1{\csname PYG@tok@#1\endcsname}
\def\PYG@toks#1+{\ifx\relax#1\empty\else%
    \PYG@tok{#1}\expandafter\PYG@toks\fi}
\def\PYG@do#1{\PYG@bc{\PYG@tc{\PYG@ul{%
    \PYG@it{\PYG@bf{\PYG@ff{#1}}}}}}}
\def\PYG#1#2{\PYG@reset\PYG@toks#1+\relax+\PYG@do{#2}}

\expandafter\def\csname PYG@tok@gd\endcsname{\def\PYG@tc##1{\textcolor[rgb]{0.63,0.00,0.00}{##1}}}
\expandafter\def\csname PYG@tok@gu\endcsname{\let\PYG@bf=\textbf\def\PYG@tc##1{\textcolor[rgb]{0.50,0.00,0.50}{##1}}}
\expandafter\def\csname PYG@tok@gt\endcsname{\def\PYG@tc##1{\textcolor[rgb]{0.00,0.27,0.87}{##1}}}
\expandafter\def\csname PYG@tok@gs\endcsname{\let\PYG@bf=\textbf}
\expandafter\def\csname PYG@tok@gr\endcsname{\def\PYG@tc##1{\textcolor[rgb]{1.00,0.00,0.00}{##1}}}
\expandafter\def\csname PYG@tok@cm\endcsname{\let\PYG@it=\textit\def\PYG@tc##1{\textcolor[rgb]{0.25,0.50,0.56}{##1}}}
\expandafter\def\csname PYG@tok@vg\endcsname{\def\PYG@tc##1{\textcolor[rgb]{0.73,0.38,0.84}{##1}}}
\expandafter\def\csname PYG@tok@m\endcsname{\def\PYG@tc##1{\textcolor[rgb]{0.13,0.50,0.31}{##1}}}
\expandafter\def\csname PYG@tok@mh\endcsname{\def\PYG@tc##1{\textcolor[rgb]{0.13,0.50,0.31}{##1}}}
\expandafter\def\csname PYG@tok@cs\endcsname{\def\PYG@tc##1{\textcolor[rgb]{0.25,0.50,0.56}{##1}}\def\PYG@bc##1{\setlength{\fboxsep}{0pt}\colorbox[rgb]{1.00,0.94,0.94}{\strut ##1}}}
\expandafter\def\csname PYG@tok@ge\endcsname{\let\PYG@it=\textit}
\expandafter\def\csname PYG@tok@vc\endcsname{\def\PYG@tc##1{\textcolor[rgb]{0.73,0.38,0.84}{##1}}}
\expandafter\def\csname PYG@tok@il\endcsname{\def\PYG@tc##1{\textcolor[rgb]{0.13,0.50,0.31}{##1}}}
\expandafter\def\csname PYG@tok@go\endcsname{\def\PYG@tc##1{\textcolor[rgb]{0.20,0.20,0.20}{##1}}}
\expandafter\def\csname PYG@tok@cp\endcsname{\def\PYG@tc##1{\textcolor[rgb]{0.00,0.44,0.13}{##1}}}
\expandafter\def\csname PYG@tok@gi\endcsname{\def\PYG@tc##1{\textcolor[rgb]{0.00,0.63,0.00}{##1}}}
\expandafter\def\csname PYG@tok@gh\endcsname{\let\PYG@bf=\textbf\def\PYG@tc##1{\textcolor[rgb]{0.00,0.00,0.50}{##1}}}
\expandafter\def\csname PYG@tok@ni\endcsname{\let\PYG@bf=\textbf\def\PYG@tc##1{\textcolor[rgb]{0.84,0.33,0.22}{##1}}}
\expandafter\def\csname PYG@tok@nl\endcsname{\let\PYG@bf=\textbf\def\PYG@tc##1{\textcolor[rgb]{0.00,0.13,0.44}{##1}}}
\expandafter\def\csname PYG@tok@nn\endcsname{\let\PYG@bf=\textbf\def\PYG@tc##1{\textcolor[rgb]{0.05,0.52,0.71}{##1}}}
\expandafter\def\csname PYG@tok@no\endcsname{\def\PYG@tc##1{\textcolor[rgb]{0.38,0.68,0.84}{##1}}}
\expandafter\def\csname PYG@tok@na\endcsname{\def\PYG@tc##1{\textcolor[rgb]{0.25,0.44,0.63}{##1}}}
\expandafter\def\csname PYG@tok@nb\endcsname{\def\PYG@tc##1{\textcolor[rgb]{0.00,0.44,0.13}{##1}}}
\expandafter\def\csname PYG@tok@nc\endcsname{\let\PYG@bf=\textbf\def\PYG@tc##1{\textcolor[rgb]{0.05,0.52,0.71}{##1}}}
\expandafter\def\csname PYG@tok@nd\endcsname{\let\PYG@bf=\textbf\def\PYG@tc##1{\textcolor[rgb]{0.33,0.33,0.33}{##1}}}
\expandafter\def\csname PYG@tok@ne\endcsname{\def\PYG@tc##1{\textcolor[rgb]{0.00,0.44,0.13}{##1}}}
\expandafter\def\csname PYG@tok@nf\endcsname{\def\PYG@tc##1{\textcolor[rgb]{0.02,0.16,0.49}{##1}}}
\expandafter\def\csname PYG@tok@si\endcsname{\let\PYG@it=\textit\def\PYG@tc##1{\textcolor[rgb]{0.44,0.63,0.82}{##1}}}
\expandafter\def\csname PYG@tok@s2\endcsname{\def\PYG@tc##1{\textcolor[rgb]{0.25,0.44,0.63}{##1}}}
\expandafter\def\csname PYG@tok@vi\endcsname{\def\PYG@tc##1{\textcolor[rgb]{0.73,0.38,0.84}{##1}}}
\expandafter\def\csname PYG@tok@nt\endcsname{\let\PYG@bf=\textbf\def\PYG@tc##1{\textcolor[rgb]{0.02,0.16,0.45}{##1}}}
\expandafter\def\csname PYG@tok@nv\endcsname{\def\PYG@tc##1{\textcolor[rgb]{0.73,0.38,0.84}{##1}}}
\expandafter\def\csname PYG@tok@s1\endcsname{\def\PYG@tc##1{\textcolor[rgb]{0.25,0.44,0.63}{##1}}}
\expandafter\def\csname PYG@tok@gp\endcsname{\let\PYG@bf=\textbf\def\PYG@tc##1{\textcolor[rgb]{0.78,0.36,0.04}{##1}}}
\expandafter\def\csname PYG@tok@sh\endcsname{\def\PYG@tc##1{\textcolor[rgb]{0.25,0.44,0.63}{##1}}}
\expandafter\def\csname PYG@tok@ow\endcsname{\let\PYG@bf=\textbf\def\PYG@tc##1{\textcolor[rgb]{0.00,0.44,0.13}{##1}}}
\expandafter\def\csname PYG@tok@sx\endcsname{\def\PYG@tc##1{\textcolor[rgb]{0.78,0.36,0.04}{##1}}}
\expandafter\def\csname PYG@tok@bp\endcsname{\def\PYG@tc##1{\textcolor[rgb]{0.00,0.44,0.13}{##1}}}
\expandafter\def\csname PYG@tok@c1\endcsname{\let\PYG@it=\textit\def\PYG@tc##1{\textcolor[rgb]{0.25,0.50,0.56}{##1}}}
\expandafter\def\csname PYG@tok@kc\endcsname{\let\PYG@bf=\textbf\def\PYG@tc##1{\textcolor[rgb]{0.00,0.44,0.13}{##1}}}
\expandafter\def\csname PYG@tok@c\endcsname{\let\PYG@it=\textit\def\PYG@tc##1{\textcolor[rgb]{0.25,0.50,0.56}{##1}}}
\expandafter\def\csname PYG@tok@mf\endcsname{\def\PYG@tc##1{\textcolor[rgb]{0.13,0.50,0.31}{##1}}}
\expandafter\def\csname PYG@tok@err\endcsname{\def\PYG@bc##1{\setlength{\fboxsep}{0pt}\fcolorbox[rgb]{1.00,0.00,0.00}{1,1,1}{\strut ##1}}}
\expandafter\def\csname PYG@tok@kd\endcsname{\let\PYG@bf=\textbf\def\PYG@tc##1{\textcolor[rgb]{0.00,0.44,0.13}{##1}}}
\expandafter\def\csname PYG@tok@ss\endcsname{\def\PYG@tc##1{\textcolor[rgb]{0.32,0.47,0.09}{##1}}}
\expandafter\def\csname PYG@tok@sr\endcsname{\def\PYG@tc##1{\textcolor[rgb]{0.14,0.33,0.53}{##1}}}
\expandafter\def\csname PYG@tok@mo\endcsname{\def\PYG@tc##1{\textcolor[rgb]{0.13,0.50,0.31}{##1}}}
\expandafter\def\csname PYG@tok@mi\endcsname{\def\PYG@tc##1{\textcolor[rgb]{0.13,0.50,0.31}{##1}}}
\expandafter\def\csname PYG@tok@kn\endcsname{\let\PYG@bf=\textbf\def\PYG@tc##1{\textcolor[rgb]{0.00,0.44,0.13}{##1}}}
\expandafter\def\csname PYG@tok@o\endcsname{\def\PYG@tc##1{\textcolor[rgb]{0.40,0.40,0.40}{##1}}}
\expandafter\def\csname PYG@tok@kr\endcsname{\let\PYG@bf=\textbf\def\PYG@tc##1{\textcolor[rgb]{0.00,0.44,0.13}{##1}}}
\expandafter\def\csname PYG@tok@s\endcsname{\def\PYG@tc##1{\textcolor[rgb]{0.25,0.44,0.63}{##1}}}
\expandafter\def\csname PYG@tok@kp\endcsname{\def\PYG@tc##1{\textcolor[rgb]{0.00,0.44,0.13}{##1}}}
\expandafter\def\csname PYG@tok@w\endcsname{\def\PYG@tc##1{\textcolor[rgb]{0.73,0.73,0.73}{##1}}}
\expandafter\def\csname PYG@tok@kt\endcsname{\def\PYG@tc##1{\textcolor[rgb]{0.56,0.13,0.00}{##1}}}
\expandafter\def\csname PYG@tok@sc\endcsname{\def\PYG@tc##1{\textcolor[rgb]{0.25,0.44,0.63}{##1}}}
\expandafter\def\csname PYG@tok@sb\endcsname{\def\PYG@tc##1{\textcolor[rgb]{0.25,0.44,0.63}{##1}}}
\expandafter\def\csname PYG@tok@k\endcsname{\let\PYG@bf=\textbf\def\PYG@tc##1{\textcolor[rgb]{0.00,0.44,0.13}{##1}}}
\expandafter\def\csname PYG@tok@se\endcsname{\let\PYG@bf=\textbf\def\PYG@tc##1{\textcolor[rgb]{0.25,0.44,0.63}{##1}}}
\expandafter\def\csname PYG@tok@sd\endcsname{\let\PYG@it=\textit\def\PYG@tc##1{\textcolor[rgb]{0.25,0.44,0.63}{##1}}}

\def\PYGZbs{\char`\\}
\def\PYGZus{\char`\_}
\def\PYGZob{\char`\{}
\def\PYGZcb{\char`\}}
\def\PYGZca{\char`\^}
\def\PYGZam{\char`\&}
\def\PYGZlt{\char`\<}
\def\PYGZgt{\char`\>}
\def\PYGZsh{\char`\#}
\def\PYGZpc{\char`\%}
\def\PYGZdl{\char`\$}
\def\PYGZhy{\char`\-}
\def\PYGZsq{\char`\'}
\def\PYGZdq{\char`\"}
\def\PYGZti{\char`\~}
% for compatibility with earlier versions
\def\PYGZat{@}
\def\PYGZlb{[}
\def\PYGZrb{]}
\makeatother

\begin{document}

\maketitle
\tableofcontents
\phantomsection\label{index::doc}


Contents:


\chapter{Pyodec Introduction}
\label{intro::doc}\label{intro:pyodec-introduction}\label{intro:welcome-to-pyodec-s-documentation}
Pyodec is intended for standardizing and sharing tools for decoding data files which cannot be efficiently read through automated methdods.


\chapter{Pyodec module methods}
\label{root::doc}\label{root:pyodec-module-methods}
Two methods are available at the root of the pyodec package. Only one of them is acutally functional
\phantomsection\label{root:module-pyodec}\index{pyodec (module)}
Pyodec root functionality
\index{decode() (in module pyodec)}

\begin{fulllineitems}
\phantomsection\label{root:pyodec.decode}\pysiglinewithargsret{\code{pyodec.}\bfcode{decode}}{\emph{source}, \emph{decoder}, \emph{*args}, \emph{**kwargs}}{}
import and execute a file or string decoder on a certain class

\end{fulllineitems}

\index{detect() (in module pyodec)}

\begin{fulllineitems}
\phantomsection\label{root:pyodec.detect}\pysiglinewithargsret{\code{pyodec.}\bfcode{detect}}{\emph{source}}{}
run every decoder we have on some amount of the source file, 
and return every decoder identifier which successfully read data from the chunk.

\end{fulllineitems}



\chapter{Pyodec core classes}
\label{classes:pyodec-core-classes}\label{classes:module-pyodec.core}\label{classes::doc}\index{pyodec.core (module)}\index{FileDecoder (class in pyodec.core)}

\begin{fulllineitems}
\phantomsection\label{classes:pyodec.core.FileDecoder}\pysiglinewithargsret{\strong{class }\code{pyodec.core.}\bfcode{FileDecoder}}{\emph{vars=False}, \emph{inherit=False}, \emph{fixed\_vars=False}}{}
The inheritable class for a decoder of files.
\index{decode() (pyodec.core.FileDecoder method)}

\begin{fulllineitems}
\phantomsection\label{classes:pyodec.core.FileDecoder.decode}\pysiglinewithargsret{\bfcode{decode}}{\emph{filepath}, \emph{generator=False}, \emph{limit=1000}, \emph{**kwargs}}{}
run the contained decode\_proc as either a generator or a procedural decoder. 
If run as a generator, generator=True.

\end{fulllineitems}

\index{decode\_chunks() (pyodec.core.FileDecoder method)}

\begin{fulllineitems}
\phantomsection\label{classes:pyodec.core.FileDecoder.decode_chunks}\pysiglinewithargsret{\bfcode{decode\_chunks}}{\emph{filepath}, \emph{limit}, \emph{begin=None}, \emph{end=None}}{}
A ``precompiled'' generator-based decoder, to allow you to skip
having to write the standard lines.

\end{fulllineitems}

\index{decode\_lines() (pyodec.core.FileDecoder method)}

\begin{fulllineitems}
\phantomsection\label{classes:pyodec.core.FileDecoder.decode_lines}\pysiglinewithargsret{\bfcode{decode\_lines}}{\emph{filepath}, \emph{limit}}{}
A precompiled generator-based decoder allowing line-decoding without having
to write the standard modules - if the default options are all that are needed.

\end{fulllineitems}

\index{decode\_proc() (pyodec.core.FileDecoder method)}

\begin{fulllineitems}
\phantomsection\label{classes:pyodec.core.FileDecoder.decode_proc}\pysiglinewithargsret{\bfcode{decode\_proc}}{\emph{filepath}, \emph{limit}, \emph{**kwargs}}{}
this should be a standardized function - defined by the decoder
which takes a file path, and opens it, and calls read\_lines or read\_chunks
and then returns the data those two functions produce.

Alternatively, you can not decode it, and use the default. But, it really won't
work for most applications. Sorry.

\end{fulllineitems}

\index{on\_chunk() (pyodec.core.FileDecoder method)}

\begin{fulllineitems}
\phantomsection\label{classes:pyodec.core.FileDecoder.on_chunk}\pysiglinewithargsret{\bfcode{on\_chunk}}{\emph{chunk}}{}
return a tuple from an observation -- defined by the specific decoder.
return False if the ob should be skipped

\end{fulllineitems}

\index{on\_line() (pyodec.core.FileDecoder method)}

\begin{fulllineitems}
\phantomsection\label{classes:pyodec.core.FileDecoder.on_line}\pysiglinewithargsret{\bfcode{on\_line}}{\emph{line}}{}
return a tuple whose indices correspond to those of varlist.
return False if the ob should be skipped

\end{fulllineitems}

\index{read\_chunks() (pyodec.core.FileDecoder method)}

\begin{fulllineitems}
\phantomsection\label{classes:pyodec.core.FileDecoder.read_chunks}\pysiglinewithargsret{\bfcode{read\_chunks}}{\emph{yieldcount}, \emph{gfhandle}, \emph{begin=False}, \emph{end=False}}{}
generator form of chunk reading

\end{fulllineitems}

\index{read\_lines() (pyodec.core.FileDecoder method)}

\begin{fulllineitems}
\phantomsection\label{classes:pyodec.core.FileDecoder.read_lines}\pysiglinewithargsret{\bfcode{read\_lines}}{\emph{yieldcount}, \emph{gfhandle}}{}
Read the file, and yield the \# of obs as a generator

\end{fulllineitems}

\index{yield\_update() (pyodec.core.FileDecoder method)}

\begin{fulllineitems}
\phantomsection\label{classes:pyodec.core.FileDecoder.yield_update}\pysiglinewithargsret{\bfcode{yield\_update}}{\emph{update}}{}
A reading process can throw updates if it wishes. The
default self.\_throw\_updates must be set to true.

\end{fulllineitems}


\end{fulllineitems}

\index{FixedVariableList (class in pyodec.core)}

\begin{fulllineitems}
\phantomsection\label{classes:pyodec.core.FixedVariableList}\pysigline{\strong{class }\code{pyodec.core.}\bfcode{FixedVariableList}}
Similar to a variable list, but much simpler, with fewer functions

\end{fulllineitems}

\index{MessageDecoder (class in pyodec.core)}

\begin{fulllineitems}
\phantomsection\label{classes:pyodec.core.MessageDecoder}\pysiglinewithargsret{\strong{class }\code{pyodec.core.}\bfcode{MessageDecoder}}{\emph{vars=False}, \emph{inherit=False}, \emph{fixed\_vars=False}}{}
Just a wrapper for the decoder class, because message decoders can (and should)
contain a varlist just as the main decoders
\index{decode() (pyodec.core.MessageDecoder method)}

\begin{fulllineitems}
\phantomsection\label{classes:pyodec.core.MessageDecoder.decode}\pysiglinewithargsret{\bfcode{decode}}{\emph{message}}{}
the decode method should be refactored, and used to decode a string message

\end{fulllineitems}


\end{fulllineitems}

\index{VariableList (class in pyodec.core)}

\begin{fulllineitems}
\phantomsection\label{classes:pyodec.core.VariableList}\pysigline{\strong{class }\code{pyodec.core.}\bfcode{VariableList}}
the requrements of the varaible list are somewhat strict, it must provide information
regarding the names of the variables, their ranges, data conversions and units.
\index{addvar() (pyodec.core.VariableList method)}

\begin{fulllineitems}
\phantomsection\label{classes:pyodec.core.VariableList.addvar}\pysiglinewithargsret{\bfcode{addvar}}{\emph{name}, \emph{longname}, \emph{dtype}, \emph{shape}, \emph{unit}, \emph{index=None}, \emph{scale=1}, \emph{offset=0}, \emph{mn=0}, \emph{mx=1}}{}
Add a variable to the variable list.

\end{fulllineitems}

\index{dtype() (pyodec.core.VariableList method)}

\begin{fulllineitems}
\phantomsection\label{classes:pyodec.core.VariableList.dtype}\pysiglinewithargsret{\bfcode{dtype}}{}{}
This utility will produce the numpy recarray dtype entry
for the pytable which will hold the data contained within.

This description could be used to create a recarray of the returned data.

To insert into pytables as a description, create the array with
np.array({[}{]},dtype=decoder.tables\_desc())

\end{fulllineitems}

\index{get\_index() (pyodec.core.VariableList method)}

\begin{fulllineitems}
\phantomsection\label{classes:pyodec.core.VariableList.get_index}\pysiglinewithargsret{\bfcode{get\_index}}{\emph{varname}}{}
return the index of the variable with the name `varname'

\end{fulllineitems}

\index{tables\_desc() (pyodec.core.VariableList method)}

\begin{fulllineitems}
\phantomsection\label{classes:pyodec.core.VariableList.tables_desc}\pysiglinewithargsret{\bfcode{tables\_desc}}{}{}
DEPRECATED: alias for self.dtype()

\end{fulllineitems}


\end{fulllineitems}



\chapter{Indices and tables}
\label{index:indices-and-tables}\begin{itemize}
\item {} 
\emph{genindex}

\item {} 
\emph{modindex}

\item {} 
\emph{search}

\end{itemize}


\renewcommand{\indexname}{Python Module Index}
\begin{theindex}
\def\bigletter#1{{\Large\sffamily#1}\nopagebreak\vspace{1mm}}
\bigletter{p}
\item {\texttt{pyodec}}, \pageref{root:module-pyodec}
\item {\texttt{pyodec.core}}, \pageref{classes:module-pyodec.core}
\end{theindex}

\renewcommand{\indexname}{Index}
\printindex
\end{document}
